\section{Постановка задачи}
\label{sec:Chapter2} \index{Chapter2}

Рассмотрим методы основанные на построении эмбеддингов. Эффективность таких сетей применительно к формированию пространства эмбеддингов существенно зависит от выбора функции потерь, поэтому проведем сравнительное исследование этих функций потерь. Для проведения такого исследования предлагается следующая постановка задачи: дан некоторый набор данных для обучения, необходимо построить сеть для генерации эмбеддингов и обучить ее с применением нескольких функций потерь. Требуется провести сравнительное исследование полученных пространств эмбеддингов с учетом их дальнейшего использования для решения задач FSL.

Этапы решения задачи:
\begin{itemize}
    \item Провести обзор наиболее популярных функций потерь в области FSL, а также реалиовать их.
    \item Выбрать метод оценки качества пространств эмбеддингов.
    \item Выбрать задачу для проведения эксперимента и модель нейронной сети.
    \item Провести сравнительный анализ пространств эмбеддингов, полученных с использованием выбранных функций потерь.
\end{itemize}