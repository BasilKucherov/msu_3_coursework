\section{Выводы}
\label{sec:Chapter6} \index{Chapter6}

\par
В рамках работы было проведено исследование подходов к повышению эффективности обучения по нескольким примерам.
На основе обзора существующих методов к решению проблем FSL для исследования был выбран метод построения эмбеддингов с использованием различных функций потерь на примере задачи распознавания ключевых слов на датасете Google Speech V2, а также выбраны метрики оценки качества получаемого пространства эмбеддингов. Для проведения исследования были реализованы функции потерь Triplet Loss, Lifted Structured Loss, N-pair loss, а также базовая нейронная сеть, цикл обучения нейронных сетей и функции вычисления метрик кластеризации. Наиболее эффективной по выбранным критериям стала функция Lifted Structures Loss, которая показала на тестовом наборе по сравнению с другими результаты лучше: до 0.48 (31\%) по критерию FC и  до 0.04 (6\%) по критерию HV.