\section{Программная реализация}
\label{sec:Chapter4} \index{Chapter4}

Для выполнения поставленной задачи использовался язык Python 3. Основной библиотекой для построения и обучения моделей стал PyTorch. В ходе работы были реализованы следующие компоненты:
\begin{itemize}
    \item Функции потерь
    \begin{itemize}
        \item Triplet Loss
        \item N-pair loss
        \item Lifted-Structured Loss
    \end{itemize}
    \item Функции для формирования пакетов (батчей) для каждой из перечисленных функций потерь
    \item Метрики качества кластеризации
    \begin{itemize}
        \item FC - feature clustering
        \item HC - hyperplane variation
    \end{itemize}
\end{itemize}

Эксперименты проводились на платформе Jupyter Notebook на локальной машине с использованием видеокарты Nvidia RTX 3060 TI 8GB.