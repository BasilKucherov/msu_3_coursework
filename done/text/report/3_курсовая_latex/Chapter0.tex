\section{Введение}
\label{sec:Chapter0} \index{Chapter0}

\par
За последнее время с помощью машинного обучения были достигнуты значительные успехи во многих областях, начиная от распознавания изображений и заканчивая автопилотами.  Обучаясь на больших наборах данных, модель способна качественно обобщать их на новые примеры. Однако сбор и маркировка таких больших датасетов может оказаться времязатратной и дорогостоящей задачей, особенно для приложений, требующих уникальных данных или связанных с конфиденциальной информацией.

Среди таких приложений можно выделить следующие:
\begin{itemize}
    \item Распознавание ключевых слов (Keyword spotting) \cite{FSLKeywordSpotting}. При использовании в системах умного дома или голосовых ассистентах для добавления новых комманд нельзя заставить пользователя наговаривать сотни раз одни те же реплики..
    % https://arxiv.org/pdf/2210.02732.pdf
    \item Верификация по биометрии \cite{FSLBiometricVerification}: распознавание отпечатков пальцев, лица или голоса владельца. Возникает аналогичная проблема малого количества экземпляров, предоставляемых пользователем.
    % https://arxiv.org/pdf/2211.06761.pdf
    \item Поиск лекарств, основанный на подборе молекул \cite{FSLDrugDiscovery}: из за сложности синтеза или возможной токсичности обычно собирается только небольшое количество реально созданных и изученных молекул - кандидатов. На основе таких кандидатов требуется подобрать безопасный и более активный аналог.
\end{itemize}

\par
С целью решения данной проблемы появился подраздел машинного обучения - обучение по нескольким примерам (Few Shot Learning или FSL \cite{FSLsurvey}). В отличие от традиционного машинного обучения, в задачах FSL обучение производится всего по нескольким примерам, что дает возможность применять нейронные сети там, где раньше это было невозможно. Поскольку стандартные алгоритмы обучения нейронных сетей в условиях недостатка примеров будут приводить к переобучению, актуальной проблемой является разработка эффективных алгоритмов для FSL.

\par
Данная работа посвящена исследованию подходов к повышению эффективности обучения нейронных сетей при решении задачи FSL. 
